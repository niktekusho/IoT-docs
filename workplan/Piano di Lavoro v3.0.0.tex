%%%%%%%%%%%%%%%%%%%%%%%%%%%%%%%%%%%%%%%%%
% NIH Grant Proposal for the Specific Aims and Research Plan Sections
% LaTeX Template
% Version 1.0 (21/10/13)
%
% This template has been downloaded from:
% http://www.LaTeXTemplates.com
%
% Original author:
% Erick Tatro (erickttr@gmail.com) with modifications by:
% Vel (vel@latextemplates.com)
%
% Adapted from:
% J. Hrabe (http://www.magalien.com/public/nih_grants_in_latex.html)
%
% License:
% CC BY-NC-SA 3.0 (http://creativecommons.org/licenses/by-nc-sa/3.0/)
%
%%%%%%%%%%%%%%%%%%%%%%%%%%%%%%%%%%%%%%%%%

%----------------------------------------------------------------------------------------
%	PACKAGES AND OTHER DOCUMENT CONFIGURATIONS
%----------------------------------------------------------------------------------------
\documentclass[a4paper]{article}
\usepackage[a4paper, headsep=1.8cm ]{geometry} % Reduce the size of the margin
%\usepackage{showframe}
% Variabili per non ripetere i contatti mille volte, MODIFICARE QUI
\newcommand{\nomeStudente}{Nicola}
\newcommand{\cognomeStudente}{Dal Maso}
\newcommand{\matricolaStudente}{1050931}
\newcommand{\emailStudente}{nicola.dalmaso.2@studenti.unipd.it}
\newcommand{\telStudente}{+39 3497212010}

\newcommand{\nomeProponente}{Tullio}
\newcommand{\cognomeProponente}{Vardanega}
\newcommand{\emailProponente}{tullio.vardanega@math.unipd.it}
\newcommand{\telProponente}{+39 0498271359}

\newcommand{\nomeTutorInterno}{Claudio}
\newcommand{\cognomeTutorInterno}{Palazzi}
\newcommand{\emailTutorInterno}{cpalazzi@math.unipd.it}
\newcommand{\telTutorInterno}{+39 0498271426}


% A note on fonts: As of 2013, NIH allows Georgia, Arial, Helvetica, and Palatino Linotype. LaTeX doesn't have Georgia or Arial built in; you can try to come up with your own solution if you wish to use those fonts. Here, Palatino & Helvetica are available, leave the font you want to use uncommented while commenting out the other one.
%\usepackage{palatino} % Palatino font
\usepackage[utf8x]{inputenc}%codifica
\usepackage[italian]{babel} % setta la lingua - necessaria per il comando /today, che altrimenti stampa in inglese
\usepackage{helvet} % Helvetica font
\usepackage{array}
\usepackage{multirow} % permette di unire più celle verticali nelle tabelle
\usepackage{hhline} % permette di utilizzare linee delle tabelle particolari
\usepackage{arydshln}
\renewcommand*\familydefault{\sfdefault} % Use the sans serif version of the font
\usepackage[T1]{fontenc}
\linespread{1.2} % A little extra line spread is better for the Palatino font
\usepackage{fancyhdr} %pacchetto per le intestazioni
\usepackage{hyperref} % pacchetto per i riferimenti
\usepackage{lipsum} % Used for inserting dummy 'Lorem ipsum' text into the template
\usepackage{amsfonts, amsmath, amsthm, amssymb} % For math fonts, symbols and environments
\usepackage{graphicx} % Required for including images
%\usepackage{booktabs} % Top and bottom rules for table
%\usepackage{wrapfig} % Allows in-line images
%\usepackage[labelfont=bf]{caption} % Make figure numbering in captions bold
\usepackage{ragged2e}
% personalizza l'intestazione e piè di pagina
\usepackage{fancyhdr}
\pagestyle{fancy}
\rhead{
	\parbox{1.7cm}{\raggedleft Università degli Studi di Padova}
	\parbox{1.5cm}{\includegraphics[height=1.5cm]{./immagini/logo-unipd.png}}	
}
\lhead{
	\parbox{10cm}{
	\nomeStudente{} \cognomeStudente \\
	\matricolaStudente\\
	Piano di lavoro stage interno
	}
}
\renewcommand{\headrulewidth}{0cm}


\hyphenation{ionto-pho-re-tic iso-tro-pic fortran} % Specifies custom hyphenation points for words or words that shouldn't be hyphenated at all

\hypersetup{
	colorlinks=true,
	linkcolor=black,
	urlcolor=blue
}

%-------------------------------------------------------------------------------------------------
%	Creato da Mich - Updated by Simone Pessotto 04/08/2015 - Updated by Beatrice Guerra 11/05/2017
%-------------------------------------------------------------------------------------------------

\begin{document}

\begin{titlepage}
	\centering
	\includegraphics[height=5cm]{./immagini/logo-unipd.png} \par \vspace{1cm}
	{\scshape\LARGE Università degli Studi di Padova \par}
	\vspace{0.5cm}
	{\scshape\Large Laurea in Informatica \par}
	\vspace{1cm}
	{\Huge\bfseries Piano di lavoro \par}
	\vspace{0.5cm}
	{\Large\itshape Stage Interno}
	\vfill
	{\scshape\Large \nomeStudente{} \cognomeStudente{}\par}
	{\scshape\large \matricolaStudente \par}
	\vfill
	{\itshape \today}
	
\end{titlepage}

\section*{Contatti}
\parbox{14.7cm}{\textbf{Studente:} \nomeStudente{} \cognomeStudente{}, \href{mailto:\emailStudente{}}{\emailStudente{}}, \telStudente{}} \\

\noindent
\parbox{14.7cm}{\textbf{Proponente:} \nomeProponente{} \cognomeProponente{}, \href{mailto:\emailProponente{}}{\emailProponente{}}, \telProponente{}} \\

\noindent
\parbox{14.7cm}{\textbf{Tutor Interno:} \nomeTutorInterno{} \cognomeTutorInterno{}, \href{mailto:\emailTutorInterno{}}{\emailTutorInterno{}}, \telTutorInterno{}} \\

\section*{Scopo dello stage}

Lo stage prevede lo sviluppo e la realizzazione di una \textit{dashboard} per la gestione di dispositivi interconnessi (IoT).\\
L'idea alla base del sistema è quella di un centro di controllo attraverso cui l'utente del sistema gestisce i dispositivi smart presenti nella propria rete domestica, permettendo operazioni del tipo:
\begin{itemize}
	\item avvio/spegnimento di un dispositivo;
	\item monitoraggio dei dispositivi collegati;
	\item richiesta di dati per conoscere lo stato dei dispositivi (es. per una lampadina: accesa/spenta, assorbimento energetico, ecc.);
	\item collegamento all'eventuale interfaccia proprietaria del dispositivo (es. supporto tecnico).
\end{itemize}
Le tecnologie impiegate per il progetto saranno:
\begin{itemize}
	\item Node.JS per i servizi lato backend;
	\item React, HTML5 e CSS3 per la realizzazione del frontend.
\end{itemize}
Data la natura altamente dinamica di questo mercato, il sistema prevede la possibilità di simulare dispositivi collegati ad esso.\\
Durante lo sviluppo del sistema, verranno impiegati sia dispositivi \textit{virtualizzati}, sia dispositivi fisici.\\
Il dispositivo fisico scelto dallo studente è il \href{https://www.raspberrypi.org/}{Raspberry Pi rev. 3} dal momento che esso consente il collegamento di numerosi tipi di sensori ed è già posseduto dallo studente.


\clearpage
\section*{Pianificazione del lavoro}
La pianificazione, in termini di quantità di ore di lavoro, sarà così distribuita:
\begin{center}	
\begin{tabular}{|>{\centering} m{1.5cm}|>{\centering} m{1.5cm}|m{10cm}|}
	\hline
	\multicolumn{2}{|c|}{\textbf{Durata in ore}} & \textbf{Descrizione dell'attività} \\
	\hline
	\multicolumn{2}{|c|}{40} & Formazione
	 \begin{itemize}
		\item Architettura microservizi
		\item Node.JS orientato ai microservizi
		\item React
	\end{itemize} 
	\\
	\hline
	
	\multirow{4}{*}{120} & & Analisi, sviluppo e implementazione servizio di comunicazione con i dispositivi IoT\\
	\cline{2-2}
	& 40 & \begin{itemize}
		\item Analisi dei protocolli open source esistenti per i diversi dispositivi IoT
		\item Stima implementazione eventuali nuovi protocolli
	\end{itemize} \\
	\cline{2-2}
	& 80 & \begin{itemize}
		\item Progettazione del servizio di comunicazione con i dispositivi
		\item Progettazione dei test del servizio di comunicazione
		\item Realizzazione dei test e del servizio di comunicazione in Node.JS
	\end{itemize} \\
	\hline
	
	\multirow{4}{*}{120} & & Analisi, sviluppo e implementazione servizio di presentazione delle informazioni agli utenti\\
	\cline{2-2}
	& 40 & \begin{itemize}
		\item Analisi interazione utente con la \textit{dashboard}
	\end{itemize} \\
	\cline{2-2}
	& 80 & \begin{itemize}
		\item Progettazione del servizio di presentazione informazioni
		\item Progettazione dei test del servizio di presentazione
		\item Realizzazione dei test e del servizio di presentazione in React, HTML5 e CSS3.
	\end{itemize} \\
	\hline
	
	\multicolumn{2}{|c|}{20} & {\textit{Review} dei servizi, \textit{deploy} dei servizi su Heroku.}\\
	\hline
	
	\multicolumn{2}{|c|}{\textbf{Totale ore}} & {\textbf{300}} \\
	\hline
	
\end{tabular}

\end{center}

\clearpage
\section*{Milestone}
In questa sezione vengono presentate le milestone previste per il progetto su base settimanale, associando a ciascuna milestone i prodotti che devono essere sviluppati entro la corrispondente scadenza.
\begin{itemize}
	\item Prima settimana: Completamento delle attività di autoformazione con produzione di una breve relazione riguardante la stessa;
	\item Seconda e terza settimana: Analisi dei protocolli di comunicazione esistenti, primo ciclo di progettazione e implementazione del servizio di comunicazione, mirato all'implementazione dei dispositivi \textit{virtualizzati};
	\item Quarta e quinta settimana: Revisione analisi sui protocolli, secondo ciclo di progettazione e implementazione del servizio di comunicazione, mirato all'implementazione dei dispositivi fisici (Raspberry Pi);
	\item Sesta e settima settimana: Analisi dell'interazione utente con la dashboard, progettazione e implementazione del servizio di presentazione;
	\item Ottava settimana: Revisione dei servizi, stesura del Manuale d'Uso e deploy (opzionale) di un ambiente di simulazione della dashboard su Heroku.
\end{itemize}

\clearpage
\section*{Obiettivi}
Si farà riferimento ai requisiti secondo le seguenti notazioni:
\begin{itemize}
	\item \textbf{Ob} per i requisiti obbligatori, vincolanti in quanto obiettivo primario richiesto dal committente;
	\item  \textbf{D} per i requisiti desiderabili, non vincolanti o strettamente necessari, ma dal riconoscibile valore aggiunto;
	\item \textbf{Op} per i requisiti opzionali, rappresentanti valore aggiunto non strettamente competitivo.
\end{itemize}
Le sigle precedentemente indicate saranno seguite da un numero, identificativo del requisito.\\
Si prevede lo svolgimento dei seguenti obiettivi:

\begin{table}[h]
	\centering
	\begin{tabular}{|>{\centering} m{2cm}|m{11cm}|}
	\hline
	\multicolumn{2}{|c|}{\textbf{Obbligatori}}\\
	\hline
	Ob1 & Realizzazione del servizio di comunicazione con i dispositivi IoT: collegamento a dispositivi \textit{virtualizzati} e a dispositivi fisici (Raspberry Pi).\\
	\hline
	Ob2 & Realizzazione del servizio di presentazione informazioni all'utente\\
	\hline
	\multicolumn{2}{|c|}{\textbf{Desiderabili}}\\
	\hline
	D1 & Realizzazione del servizio di autenticazione per l'accesso ai servizi\\
	\hline
	\multicolumn{2}{|c|}{\textbf{Opzionali}}\\
	\hline
	Op1 & \textit{Deploy} dei servizi su Heroku\\
	Op2 & Interazione dei servizi con \textit{database} per funzionalità di monitoring\\
	\hline
\end{tabular}

\end{table}

\subsection*{Obiettivi formativi}
L'obiettivo di questo stage è quello di approfondire, attraverso la realizzazione di un sistema concreto, il pattern architetturale a \textbf{microservizi}.\\
L'implementazione dei servizi del sistema è basata sulla piattaforma \href{https://nodejs.org/it/}{Node.JS}. Node.JS è un framework già conosciuto dallo studente, utilizzato da alcune tra le più grandi aziende del mondo ITC e il suo crescente adottazione da parte del pubblico lo rende una competenza estremamente interessante per lo studente.\\
Ad accrescere l'interesse verso il progetto oggetto dello stage è l'ambito tecnologico denominato IoT (\href{https://en.wikipedia.org/wiki/Internet_of_things}{Internet of Things}). L'automazazione di sempre più numerosi aspetti della nostra vita quotidiana, dalla spesa alle pulizie domestiche, rende questo argomento estremamente interessante per lo studente.

\clearpage
\section*{Prodotti dello Stage}
Le attività dello Stage portano alla analisi, progettazione e realizzazione del sistema in oggetto del progetto.\\
Le attività di analisi e progettazione richiedono la produzione dei seguenti documenti:
\begin{itemize}
	\item Analisi dei Requisiti, contenente: \begin{itemize}
		\item scopo del progetto;
		\item descrizione del progetto;
		\item requisiti del sistema dal punto di vista funzionale e strutturale.
	\end{itemize}
	\item Specifica Tecnica, contentente: \begin{itemize}
		\item tecnologie impiegate nel progetto;
		\item descrizione dell'architettura del sistema;
		\item definizione dell'architettura del sistema, mediante l'utilizzo di diagrammi UML.
	\end{itemize}
\end{itemize}
Le attività di implementazione richiederanno la produzione di codice sorgente pubblicamente disponibile su \href{https://github.com}{GitHub}.
Lo studente si impegna inoltre ad allegare al codice sorgente un Manuale d'Uso del prodotto, contentente istruzioni per: \begin{itemize}
	\item ottenere il codice sorgente;
	\item eseguire il codice ottenuto;
	\item testare il codice ottenuto.
\end{itemize}
Il Manuale d'Uso sarà proposto nella pagina \textit{README} visualizzata di default da GitHub quando un utente naviga verso l'indirizzo del repository del progetto.

\clearpage
\section*{Fonti informative}
Le attività di autoformazione si basano sullo studio e la consultazione dei seguenti contenuti tecnici/informativi:
\begin{itemize}
	\item \href{https://martinfowler.com/articles/microservices.html}{Martin Fowler - Microservices, a definition of this new architectural term}
	\item \href{https://en.wikipedia.org/wiki/Microservices}{Microservices - Wikipedia}
	\item \href{https://medium.com/@cramirez92/build-a-nodejs-cinema-microservice-and-deploying-it-with-docker-part-1}{Sample per la progettazione e implementazione di microservizi attraverso Node.JS}
	
	\item \href{https://reactjs.org/docs/}{Documentazione ufficiale React}
	
	\item \href{https://github.com/real-logic/aeron/wiki/Protocol-Specification}{Aeron protocol specification}
	
	\item \href{https://github.com/mqtt/mqtt.github.io/wiki}{MQTT protocol}
	
	\item \href{https://websocket.org/}{WebSocket protocol}
	
\end{itemize}



\end{document}
